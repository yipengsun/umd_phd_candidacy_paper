Leptons participate in electroweak interaction only;
the interaction can be described by a Lagrangian of the following form:
$\Lden{ew} = \Lden{gauge} + \Lden{f} + \Lden{\phi} + \Lden{Yuk}$.
The fermion part of \Lden{ew} reads \cite{Langacker:2010zza}:
\begin{align*}
    \Lden{f} = \sum_{l = 1}^F \Big(
        & \bar{\spin{q}}^0_{lL} i \fsl{D} \spin{q}^0_{lL} +
          \bar{\spin{l}}^0_{lL} i \fsl{D} \spin{l}^0_{lL} + \\
        & \bar{u}^0_{lR} i \fsl{D} u^0_{lR} +
          \bar{d}^0_{lR} i \fsl{D} d^0_{lR} +
          \bar{e}^0_{lR} i \fsl{D} e^0_{lR} +
          \bar{\nu}^0_{lR} i \fsl{D} \nu^0_{lR}
    \Big)
\end{align*}
where the number $F$, empirically 3, of fermion flavors is summed over, and
$L,R$ denote $SU(2)_L$ doublet\footnote{
    The left-handed lepton doublet is defined as:
    $\spin{l}^0_{lL} = \begin{pmatrix} \nu_l \\ l \end{pmatrix}$,
    where $l$ denotes lepton flavor.
}
and singlet in each flavor generation.
From the Lagrangian we see that the interactions between fermions and gauge
bosons (the interactions are embedded in the \fsl{D} operator) is independent
of their flavor.
But this is only an incomplete picture:
Fermions acquire their mass through their interaction with gauge bosons, which
is omitted above.

No mass term of the form $m \overline{\Psi} \Psi$ is permitted, since it would
spoil $SU(2)$ symmetry of the Lagrangian\footnote{
    To be precise, \Lden{ew} is locally invariant under the transformations in
    $SU(2)_L \otimes U(1)$ group.
}.
Instead, we add a doublet scalar field $\Phi$ interacting on both gauge bosons
and fermions.
After spontaneous symmetry breaking of the vacuum state, the Lagrangian remains
unbroken, and the new terms in the Lagrangian are interpreted as mass terms.
We inspect the mass terms for the leptons \cite{Langacker:2010zza}\footnote{
    The notation has been simplified.
}:
\begin{equation*}
    m_l \equiv \Gamma_l \frac{\nu}{\sqrt{2}}
\end{equation*}
with $\nu$ interpreted as vacuum expectation value of $\Phi$.
This shows that leptons coupling stronger to the $\Phi$ (Higgs) field will be
more massive.

This completes the picture.
Now we see that SM demands LFU, except for the Higgs coupling.
