However, the above picture (\autoref{sec:lfu}) is incomplete:
leptons acquire their mass through their interaction with the Higgs bosons,
but different flavor of leptons have different mass, this implies lepton-Higgs
coupling varies with flavor.

The electroweak Lagrangian is locally invariant under $SU(2)$ transformations;
this symmetry needs to be preserved.
Naive mass terms of the form $\bar{\Psi} m \Psi$
would spoil $SU(2)$ symmetry of the Lagrangian, hence they are not permitted.
Instead, we add a doublet scalar field $\Phi$, known as Higgs doublet,
interacting with both gauge bosons and fermions.
After spontaneous symmetry breaking of the vacuum state, the Lagrangian
preserves its symmetry, and the new terms in the Lagrangian, encapsulated in
\Lden{yuk}, are interpreted as mass terms.

We inspect the coupling in the mass terms for the
leptons \cite{Langacker:2010zza}\footnote{
    The notation has been simplified.
}
\begin{equation}
    m_l \equiv \Gamma_l \frac{\nu}{\sqrt{2}},
\end{equation}
with $\nu$ as the vacuum expectation value of $\Phi$, and $\Gamma_l$ the
coupling between lepton with flavor $\ell$ and Higgs.
Because the mass of a lepton is directly proportional to the lepton-Higgs
coupling, lepton flavors with a stronger coupling to the Higgs will be more
massive.

The Higgs mechanism completes the picture:
Now we see that SM demands LFU, except for the Higgs coupling.
