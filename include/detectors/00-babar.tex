% Talk about PEP-II and its asymmetrical beam energies
PEP-II is an asymmetrical $e^- e^+$ collider at SLAC.
In PEP-II, $B$ mesons are produced exclusively in the following process:
$e^- e^+ \rightarrow \Y4S/ \rightarrow B \overline{B}$, with 
$e^-$ and $e^+$ beams tuned at different energies,
such that the invariant mass is at \Y4S/ resonance,
and the momentum of the \Y4S/ in the lab frame
non-zero \cite{Harrison:1998yr}.

Producing at \Y4S/ peak eliminates almost all fragmentation products, reducing combinatorial
background.
Also, since the momenta of $e^- e^+$ is known, with the reconstruction of the
momentum of one $B$ meson ($B_{tag}$), the rest frame of the other $B$ ($B_{sig}$) can 
be calculated as \cite{Harrison:1998yr}
\begin{equation}
    p_{B_{sig}} = p_{e^-e^+} - p_{B_{tag}}.
\end{equation}
Later we will see that this makes identifying events that have more than one missing particles
($\mu$ decay) easier.

% Talk about subdetectors
\BaBar/ is a barrel detector (shown in \autoref{fig:babar_detector_view})
consists of five subdetectors (from inside out):
Silicon Vertex Tracker (SVT) and Drift Chamber (DCH), which measure the momenta
and angles of charged particles.
Detector of Internally Reflected Cerenkov radiation (DIRC), together with SVT
and DCH, identifies charged particles of different masses by Cerenkov ring-
imaging and ionization energy loss of these particles.
Caesium Iodide Electromagnetic Calorimeter (EMC), which measures energy and
position of electromagnetic showers generated by electrons and photons.
A superconducting solenoid with a \SI{1.5}{T} magnetic field surrounding the
EMC, together with Instrumented Flux Return (IFR), is used to identify muons and
some neutral hadrons \cite{Lees:2013uzd}.

\begin{figure}[ht]
    \centering
    \includegraphics[width=0.7\textwidth]{figs/babar_detector_view.pdf}
    \caption{
        View of the \BaBar/ detector.
    }
    \label{fig:babar_detector_view}
\end{figure}

% Talk about BaBar being 4 pi
The distribution of angular cross section is less
polarized \cite{Boutigny:1995ib,McGregor:2008ek}, thus the detector needs to
cover almost all solid angles (a $4\pi$ detector).
Indeed, \BaBar/ has tracking coverage of 0.92, namely 92\% of the $4\pi$ solid
angle \cite{Harrison:1998yr}.

% Talk about tracking and calorimeters
$B$ physics requires excellent vertex resolution and tracking, because the two
$B$ mesons produced by \Y4S/ must be reliably separated.
\BaBar/ has excellent tracking for charged particles, and sufficient spatial
and energy resolution in the electromagnetic calorimeter to reconstruct the momenta of neutral
particles \cite{Harrison:1998yr,Bauer:2005} with adequate precision.

% Talk about luminosity
\BaBar/ collected data from 1999 to 2008. Its integrated luminosity for \Y4S/
reached $424.18 \pm 0.04 \pm 1.82$~\si{fb^{-1}}.
This resulted in $(464.8 \pm 2.8) \times 10^6$ $B \overline{B}$
events \cite{Lees:2013rw}.
