The $B$ factories reconstruct $B \rightarrow D^{(*)} \tau^- \overline{\nu}_\tau$
(signal) and $B \rightarrow D^{(*)} \ell^- \overline{\nu}_\ell$ (normalization)
decays by selecting events with a tagged $B$ meson ($B_{tag}$), a $D^{(*)}$
meson, and an $e$ or $\mu$.
As described in \autoref{sec:babar}, the $B$ factories can estimate the momenta
of the $B \overline{B}$ system precisely by tagging and reconstructing the other
$B$, known as $B_{tag}$.
By subtracting the momentum of $B_{tag}$ to that of initial $e^+e^-$ system, the
momentum of the signal $B_{sig}$, which decays semileptonically and thus
contains unreconstructed neutrinos in the final state, can be inferred.

\BaBar/ and BELLE independently developed two types of tagging algorithms:
semileptonic tagging and hadronic tagging.
Semileptonic tagging finds $B_{tag}$ with the following decay:
$B^- \rightarrow D^{(*)} \ell^- \bar{\nu}_\ell$, where $\ell$ is a $e$ or $\mu$.
This has the advantage of a larger branching ratio, thus more ($\approx 1\%$)
\Y4S/ events are tagged.
However, in this type of events, the $B_{tag}$ side has at least one missing
neutrino, which makes the reconstruction of $p_{B_{sig}}$ less
precise \cite{Ciezarek:2017yzh}.

On the other hand, hadronic tagging searches over a very large number of
hadronic decay chains of $B$ for each \Y4S/ event, tagging the ones that match
one of the known modes.
This has a smaller tagging rate ($\approx 0.3\%$), but because $B_{tag}$ decays
hadronically, no missing neutrino is present in the tagged final product.
This makes the reconstructed momentum of $B_{sig}$ very
precise \cite{Lees:2013uzd,Ciezarek:2017yzh}.

$D^{0}$ and $D^{+}$ mesons are reconstructed in numerous final states, including
final states with neutral particles.
$D^{*0}$ and $D^{*+}$ mesons are reconstructed by associating soft pions or
photons with previously reconstructed $D$ mesons.

All $B$ factory measurements choose a particular decay mode of the $\tau$:
$\tau^- \rightarrow \ell^- \nu_\tau \bar{\nu}_\ell$.
In this way, the signal and the normalization
$B \rightarrow D^{(*)} \ell^- \overline{\nu}_\ell$ decays are reconstructed in
the same final state, leading to the cancellation of several sources of
experimental uncertainty in the \RDDst/ ratios.
While both signal and normalization events have the same visible particles in
the final state, the signal mode has three neutrinos in the final product,
whereas the normalization mode has only one.
By looking at the missing mass of the $B_{sig}$, defined as
\begin{equation}
    m^2_{miss} \equiv \left(p_{B_{sig}} - p_{visible}\right)^2,
\end{equation}
the signal, which has a non-zero $m^2_{miss}$, can be readily differentiated
from the normalization, which does have a $m^2_{miss} \approx 0$.

Non-$B \overline{B}$ background and misreconstructed events are suppressed by
rejecting events with tracks that are not used in the reconstruction of the
$B_{tag}$, $D^{(*)}$, or lepton \cite{Ciezarek:2017yzh}.

The main background remaining is due to semileptonic $B \rightarrow D^{**} \ell
\bar{\nu}_\ell$ decays.
$D^{**}$ decays into a $D^{*}$ and a number of soft pions, which are often not
reconstructed.
As a result, the $m^2_{miss}$ distribution is similar to that of the signal.
This background is constrained by constructing $D^{(*)}\pi^0\ell$ control
samples with the same selection as the signal samples plus an additional
$\pi^0$.
In these control samples, $D^{**}$ mesons that decay to $D^{(*)} \pi^0$ are
fully reconstructed, leading to an easily distinguishable peak in the
$m^2_{miss}$ distribution \cite{Lees:2013uzd}.

The signal and normalization yields are extracted from maximum-likelihood fits,
which rely primarily on the $m^2_{miss}$ and $|\vec{p}^*_\ell|$ distributions.
The fit result is shown in \autoref{fig:babar_lhcb_fit_comparison}.
The probability distribution functions for all contributions are taken from
Monte-Carlo simulated samples, with corrections coming from data control
samples.

As can be seen in \autoref{tab:results}, all four $B$ factory results are
limited by the size of their data samples.
