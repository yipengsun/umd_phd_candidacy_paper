\subsection{2013 \BaBar/ \RDDst/}
This paper \cite{Lees:2013uzd} is the first measurement that observers an excess
in the \RD/ and \RDst/ ratio:
There is a $2.0 \sigma$ and $2.7 \sigma$ deviation from the SM, for \RD/ and
\RDst/ respectively, and a combined $3.4 \sigma$ deviation.
I will review some of the techniques applied in this analysis.

The signature of a semileptonic $B$ decay into $\tau$, such as
$B^0 \longrightarrow D^{-(*)} (\longrightarrow \text{various modes})
\tau^+ (\longrightarrow \mu^+ \nu_{\mu} \bar{\nu}_{\tau}) \nu_{\tau}$, is the
non-zero missing mass:
Since $\tau$ has a very short lifetime, it decays into a $\mu$ and two
neutrinos;
this results in a 3-neutrino final state.
Now, the missing mass $m_{miss}$ is defined to be:
\begin{equation*}
    m_{miss} \equiv \left(p_{B_{sig}} - p_{visible}\right)^2
\end{equation*}
if there is only one neutrino, then $m_{miss} = m_{\nu} \approx 0$;
on the other hand, three neutrinos very frequently gives non-zero $m_{miss}$.

Hence, it is crucial to find $p_{B_{sig}}$.
In a $e^- e^+$, collider, the invariant mass is known.
As long as we find the momentum of the other $B$ meson, denoted as
$p_{B_{tag}}$, we know $p_{B_{sig}}$;
that is, we tag the decay of the \Y4S/:
\begin{equation*}
    \Y4S/ \longrightarrow B \overline{B} \xrightarrow{\text{tagging}}
        B_{tag} B_{sig}
\end{equation*}

\BaBar/ and BELLE independently developed two types of tagging algorithms:
semileptonic tagging and hadronic tagging.
Semileptonic tagging finds $B_{tag}$ with the following decay:
$B^+ \longrightarrow l^+ \nu_l$, where $l+$ is $e^+ or \mu^+$.
This has the advantage of a larger branching ratio, thus more ($\approx 1\%$)
\Y4S/ events are tagged.
However, in this type of events, at least two neutrinos are presents, which
makes the reconstruction of $p_{B_{sig}}$ less precise \cite{Ciezarek:2017yzh}.

In this paper, hadronic tagging algorithm is improved and used:
It listed a very large number of hadronic decay chains of $B$;
for each \Y4S/ event, it compares the decay product of each $B$, tagging the
ones that match one of the listed modes.
This has a smaller tagging rate ($\approx 0.3$), but because the tagging side
momentum is reconstructed precisely (no neutrino, so all hadronic particles are
in principle reconstructed), the estimation on $p_{B_{sig}}$ is better, which
leads to a more precise measurement on
$m_{miss}$ \cite{Lees:2013uzd,Ciezarek:2017yzh}.

Another interesting technique is the usage of Gaussian non-parametric kernel
estimators in the fit.
This method has the added benefit of knowing the exact relation between variance
and bias, making optimization easier.
After numerous validation processes, it is concluded that the kernel estimators
performed well \cite{Lees:2013uzd}.

% BELLE measurements
BELLE experiment at KEK measured semileptonic $B$ decays with $\tau$ decay both
leptonically and hadronically.
Both semileptonic and hadronic tagging were used for the leptonic $\tau$ decay;
for hadronic $\tau$ decay, only hadronic tag was used.
The overall \RDDst/ deviation from the SM is about
$2\sigma$ \cite{Hirose:2017185}.
