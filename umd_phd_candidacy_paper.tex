\documentclass[12pt,letterpaper]{article}

%{{{ Package configuration
%%%%%%%%%%%%%%%%%%%%%%%%%%%%%%%%%%%%%%%%%%%%%%%%%%%%%%%%%%%%%%%%%%%%%%%%%%%%%%%%

% Page margin
\usepackage[margin=1in]{geometry}

% Support for bold small cap font
\usepackage[tuenc]{fontspec}
\setmainfont[
    Path=./fonts/,
    Extension=.otf,
    BoldFont=cmu-serif-bold,
    BoldItalicFont=cmu-serif-bold-italic,
    ItalicFont=cmu-serif-italic,
]{cmu-serif}

% Better typesetting quality
\usepackage{microtype}

% Make letter spacing work for both XeLaTeX and LuaLaTeX
\usepackage{ifluatex}
\ifluatex
    \newcommand{\LSSTYLE}{\lsstyle}
\else
    \newcommand{\LSSTYLE}{\addfontfeatures{LetterSpace=12}}
\fi

% Math
\usepackage{amsmath}
\renewcommand{\vec}[1]{\mathbf{#1}}                   % Bold as vector
\newcommand{\spin}[1]{#1}                             % spinor doing nothing
\newcommand{\Lden}[1]{\ensuremath{\mathcal{L}_{#1}}}  % Lagrangian density

% SI units
\usepackage{siunitx}

% Figure
\usepackage{float,graphicx}

% Feynman slash notation
\usepackage{centernot}
\newcommand{\fsl}[1]{\ensuremath{\centernot{#1}}}

% Redefine section, subsection styles
\usepackage[compact,center,explicit]{titlesec}
\usepackage{textcase}
\titleformat{\section}{\LSSTYLE\normalsize\scshape\filcenter}
    {\thesection}{1em}{\MakeTextUppercase{#1}}
\titleformat{\subsection}{\small\bfseries\filcenter}
    {\thesubsection}{1em}{#1}

% PRL-style horizontal rule
\usepackage{amssymb}
\newcommand{\PRLrule}{
    \bigskip
    \noindent\makebox[\linewidth]{
        \resizebox{0.3333\linewidth}{1pt}{$\blacklozenge$}
    }
    \bigskip
}

% Bold math in section title
\makeatletter
\g@addto@macro\bfseries{\boldmath}
\makeatother

% Set up author affiliation
\usepackage[affil-it]{authblk}

% Set up link, with (hopefully) math symbol support
\usepackage[pdfencoding=auto,psdextra]{hyperref}
\hypersetup{colorlinks,breaklinks,citecolor=blue}
\usepackage{cleveref}

% Set up bibliography
\usepackage[
    %style=phys,
    giveninits=true,
    %backref=true,
    natbib=true,
    backend=biber,
    doi=true,
    % Sort by the order of citation
    sorting=none,
    % This options ensures that no automatic et al. is generated
    %maxbibnames=99,
    % This option must be enabled with 'babel' package
    useprefix=false
]{biblatex}
\addbibresource{umd_phd_candidacy_paper.bib}

%}}}

%{{{ User settings
%%%%%%%%%%%%%%%%%%%%%%%%%%%%%%%%%%%%%%%%%%%%%%%%%%%%%%%%%%%%%%%%%%%%%%%%%%%%%%%%

% User-defined variables
\def\BaBar/{\textsc{BaBar}}
\def\Y4S/{\ensuremath{\Upsilon(\text{4S})}}
\def\RD/{\ensuremath{\mathcal{R}(D)}}
\def\RDst/{\ensuremath{\mathcal{R}(D^{*})}}
\def\RDDst/{\ensuremath{\mathcal{R}(D^{(*)})}}

% Title info
\title{Review on testing lepton flavor universality in semileptonic channels}
\author{Yipeng Sun}
\affil{Department of Physics, University of Maryland}
\date{\today}

%}}}

\begin{document}
\maketitle

\begin{abstract}
    What is LFUV? (FU physics, FUV physics)
    Why semileptonic decays (good thing about form factor)?
    A historical review, starting from 2012 BaBar;
    Talk about Phoebe's run 1 analyses.
    Talk about what we are doing.
    Talk about expected improvements and drawbacks of our upcoming analysis compared
    to run 1.
\end{abstract}

\section{Introduction}
The Standard Model (SM) has been very successful in describing the interactions
between elementary particles such as quarks and leptons.
The theory has been tested experimentally to a very high precision.
However, there are phenomena that cannot be explained by the SM, such as
the observed matter-antimatter asymmetry in the universe, hinting that there might be New Physics (NP) beyond the SM.
One way to search for NP is to measure the decay rates of certain processes
very precisely;
rates that differ from the SM predictions may provide evidence for NP.

Through experimental discovery, it has been established that leptons have three
flavors:
Charged leptons, namely electron $e$, muon $\mu$, and tau $\tau$;
their corresponding charge-neutral neutrinos: $\nu_e$, $\nu_\mu$ and $\nu_\tau$.
SM assumes that all three flavors of leptons participate in all
interactions with the same strength, except for the Higgs mechanism through which
they acquire their mass.
This is known as lepton flavor universality (LFU).

LFU has been tested in many precision measurements, such as the decay rate
of $K^- \rightarrow e^- \nu_e$ versus $K^- \rightarrow \mu^- \nu_\mu$\footnote{
    Unless specified, charge conjugation is assumed throughout the paper.
}.
So far, no definite violation of LFU has been detected.

In this paper, we focus on the semileptonic decays of $B$ mesons, such as
$B^- \rightarrow D^{(*)} l^- \overline{\nu}_l$:
Currently, the world average of the combined decay rate ratio $\RDDst/$,
defined as\footnote{
    $\mathcal{B}$ denotes branching fraction.
}:
\begin{equation}
    \RDDst/ \equiv \frac{
        \mathcal{B}\left(
            B^{0,-} \rightarrow D^{+,0(*)} \tau^- \overline{\nu}_\tau
        \right)
    }{
        \mathcal{B}\left(
            B^{0,-} \rightarrow D^{+,0(*)} \mu^- \overline{\nu}_\mu
        \right)
    }
\end{equation}
has a $3.08\sigma$ deviation from the SM prediction \cite{HFLAV:2019}, pointing
to a possible lepton flavor universality violation (LFUV);
many collaborations are working on more precise measurements to provide a
definite answer.

In this paper, we begin with a theoretical review on
why SM manifests LFU;
extensions to SM, such as 2-Higgs doublet model (2HDM), to permit LFUV;
and advantages of using semileptonic channels for this type of measurements.
We then review and compare colliders and detectors, such as \BaBar/ at PEP-II
and LHCb at the Large Hadron Collider (LHC), used in the testing of LFU.
After that, we will review current experimental results.
Finally, we provide an overlook on prospect for updating the $\RDDst/$ measurement with LHCb Run
2 data.

\section{Theory}
\subsection{Lepton flavor universality}
Leptons participate in electroweak interaction only;
the interaction can be described by a Lagrangian of the following form:
$\Lden{ew} = \Lden{gauge} + \Lden{f} + \Lden{\phi} + \Lden{Yuk}$.
The fermion part of \Lden{ew} reads \cite{Langacker:2010zza}:
\begin{align*}
    \Lden{f} = \sum_{l = 1}^F \Big(
        & \bar{\spin{q}}^0_{lL} i \fsl{D} \spin{q}^0_{lL} +
          \bar{\spin{l}}^0_{lL} i \fsl{D} \spin{l}^0_{lL} + \\
        & \bar{u}^0_{lR} i \fsl{D} u^0_{lR} +
          \bar{d}^0_{lR} i \fsl{D} d^0_{lR} +
          \bar{e}^0_{lR} i \fsl{D} e^0_{lR} +
          \bar{\nu}^0_{lR} i \fsl{D} \nu^0_{lR}
    \Big)
\end{align*}
where the number $F$, empirically 3, of fermion flavors is summed over, and
$L,R$ denote $SU(2)_L$ doublet\footnote{
    The left-handed lepton doublet is defined as:
    $\spin{l}^0_{lL} = \begin{pmatrix} \nu_l \\ l \end{pmatrix}$,
    where $l$ denotes lepton flavor.
}
and singlet in each flavor generation.
From the Lagrangian we see that the interactions between fermions and gauge
bosons (the interactions are embedded in the \fsl{D} operator) is independent
of their flavor.
But this is only an incomplete picture:
Fermions acquire their mass through their interaction with gauge bosons, which
is omitted above.

No mass term of the form $m \overline{\Psi} \Psi$ is permitted, since they
spoil $SU(2)$ symmetry, which is required by the electroweak
Lagrangian\footnote{
    To be precise, \Lden{ew} is locally invariant under the transformations in
    $SU(2)_L \otimes U(1)$ group.
}.
Instead, we add a doublet scalar field $\Phi$ interacting on both gauge bosons
and fermions.
After spontaneous symmetry breaking, the new terms in the Lagrangian are
interpreted as mass terms.
We inspect the mass terms for the leptons \cite{Langacker:2010zza}\footnote{
    The notation has been simplified.
}:
\begin{equation*}
    m_l \equiv \Gamma_l \frac{\nu}{\sqrt{2}}
\end{equation*}
with $\nu$ interpreted as vacuum expectation value of $\Phi$.
This shows that leptons coupling stronger to the $\Phi$ (Higgs) field will be
more massive.

This completes the picture.
Now we see that SM demands LFU, except for the Higgs coupling.


\subsection{2-Higgs doublet models}
% type-II model claimed to be excluded; type-III (very similar to type-II) still
% alive.

\subsection{Leptoquark models}

\subsection{Advantages of semileptonic channel decays}

\section{Review of colliders/detectors used in testing LFU}
\subsection{\BaBar/ at PEP-II} \label{sec:babar}
% Talk about PEP-II and its asymmetrical beam energies
PEP-II is an asymmetrical $e^- e^+$ collider at SLAC.
In PEP-II, $B$ mesons are produced primarily in the following process:
$e^- e^+ \rightarrow \Y4S/ \rightarrow B \bar{B}$, with
$e^-$ and $e^+$ beams tuned at different energies,
such that the invariant mass is at the \Y4S/ resonance (\SI{10.58}{GeV}),
and the momentum of the \Y4S/ in the lab frame
non-zero \cite{Harrison:1998yr}.

Producing at \Y4S/ peak leads to near-exclusive $B \bar{B}$ pair meson
production, reducing combinatorial background.
Also, since the momenta of $e^- e^+$ is known, with the reconstruction of the
momentum of one $B$ meson ($B_{tag}$), the momentum of the other $B$
($B_{sig}$) can be calculated as
\begin{equation}
    p_{B_{sig}} = p_{e^-e^+} - p_{B_{tag}}.
\end{equation}
Later we will see that this makes identifying events that have more than one
missing particle easier.

% Talk about subdetectors
\BaBar/ is a nearly $4\pi$ spetrometer (shown in
\autoref{fig:babar_detector_view}) that consists of five subdetectors.
From inside out:
Silicon Vertex Tracker (SVT) and Drift Chamber (DCH), which measure the momenta
and angles of charged particles with the help of a \SI{1.5}{T}.
Detector of Internally Reflected Cerenkov radiation (DIRC), together with SVT
and DCH, identifies charged particles of different masses by Cerenkov
ring-imaging and ionization energy loss of these particles.
Cesium Iodide Electromagnetic Calorimeter (EMC) measures energy and
position of electromagnetic showers generated by electrons and photons.
A superconducting solenoid with a \SI{1.5}{T} magnetic field surrounding the
EMC is part of the tracking and particle identification system which identifies
muons and some neutral hadrons together with Instrumented Flux Return
(IFR) \cite{Lees:2013uzd}.

\begin{figure}[ht]
    \centering
    \includegraphics[width=0.7\textwidth]{figs/babar_detector_view.pdf}
    \caption{
        View of the \BaBar/ detector.
        Extracted from \cite{Boutigny:1995ib}.
    }
    \label{fig:babar_detector_view}
\end{figure}

% Talk about BaBar being 4 pi
At $B$ factories, $b \bar{b}$ are produced at all angles with
non-negligible probability \cite{Boutigny:1995ib,McGregor:2008ek}, thus the
detector needs to cover almost all solid angles (a $4\pi$ detector).
Indeed, \BaBar/ has tracking coverage of 0.92, namely 92\% of the $4\pi$ solid
angle.

% Talk about tracking and calorimeters
Measurement of time dependent CP violation in neutral $B$ decays requires
excellent vertex resolution and tracking, because the two $B$ mesons produced by
\Y4S/ must be reliably separated.
\BaBar/ has excellent tracking for charged particles, and sufficient spatial
and energy resolution in the electromagnetic calorimeter to reconstruct the
momenta of neutral particles \cite{Bauer:2005} with good precision.


\subsection{LHCb at the LHC} \label{sec:lhcb}
The LHC is a $pp$ collider.
% Talk about the LHC being a hadron collider and the difficulties associated
% with it
Unlike electron, proton is a composite particle, made of $u, u, d$ quarks.
From the parton distribution functions of LHCb\footnote{
    Number density to find fraction of the momentum (denoted as $x$) at certain
    squared energy scale $Q^2$.
}, we see that there are plenty of other elementary particles, such as gluons,
that can participate in the collision.
These particles may carry varying portion of the total
momentum \cite{Ball:2014uwa}.

Effectively, partons, including gluons, are being collided.
Due to the unavoidable strong actions, many unwanted particles will be
generated---Comparing to \BaBar/, one notable addition of LHCb is the triggering
system, which is used to filter out uninteresting events, reducing the readout
rate \cite{LHCb:2008}.
Also, since we do not know the precise fraction of momentum carried by
interactive partons\footnote{
Again, these are characterized by parton distribution functions}, the $B$ meson
rest frame is not readily known.

It is obvious that $e^- e^+$ colliders provide a \emph{much} cleaner background.
However, the LHC generates much more $b\bar{b}$\footnote{
    Note that this is \emph{not} necessarily \Y4S/.
} events compared to \BaBar/, due to a much larger cross section.
At \SI{13}{TeV}, the measured cross section at LHCb\footnote{
    For $2 < \eta < 5$ only, since this is the LHCb acceptance range.
} is $144 \pm 1 \pm 21$~\si{\mu b} \cite{Aaij:2016avz}.
% FIXME: Is the calculation for BaBar cross section legal?
Use the integrated luminosity and total number of $B\overline{B}$ events
contained in the on-resonance \Y4S/ sample, we compute the $b\bar{b}$ cross
section of \BaBar/ to be $\approx 1.09$~\si{nb}, which is much smaller than that
of the LHCb.

% Talk about subdetectors
LHCb, a single-arm spectrometer, is one of the four large experiments at the
LHC.
Its constituent subdetectors, from closest to farthest from the collision point,
are shown in \autoref{fig:lhcb_detector_view}:
The Vertex Locator (VELO) provides precise measurements of track coordinates
close to the collision point.
Two Ring Imaging Cerencov counters (RICH1, RICH2) provide particle
identification for charged particles over a wide range of momentum.
Tracker Turicensis (TT) and Inner Tracker (IT) provide additional tracking for
charged particles.
The Outer Tracker (OT) is used for tracking, as well as measures the momentum
of charged particles.
The calorimeters (ECAL and HCAL) have a first-level (L0) trigger to select
hadron, electron, and photon candidates based on their transverse momentum
$p_T$;
they also provide identification for the particles listed above;
finally, they provide energy and position measurements for these particles.
The Muon system (M1-5) is farthest from the collision point;
it provides L0 high $p_T$ muon trigger, and a high-level trigger (HLT) for muon
identification \cite{LHCb:2008}.

\begin{figure}[ht]
    \centering
    \includegraphics[width=0.7\textwidth]{figs/lhcb_detector_view.pdf}
    \caption{
        View of the LHCb detector.
        Vertex Locator (VELO) is closest to collision point.
    }
    \label{fig:lhcb_detector_view}
\end{figure}

% Talk about LHCb being forward-only
An interesting design choice is the geometry of the LHCb detector:
Instead of being a barrel $4\pi$ detector, it is forward-only.
This is because at high energies, $b\bar{b}$ is mostly produced in the forward
and backward direction.
The LHCb design is a very cost-effective way to construct a detector at the LHC
dedicated for $B$ physics \cite{LHCb:2008}.

% Talk about tracking
LHCb has a very good vertexing and tracking system, some of the subdetectors has
a better resolution, even compared to \BaBar/;
but its calorimeters are mediocre \cite{LHCb:2008,Guz:2017}, which makes the
reconstruction of charge-neutral particles, such as $\pi^0$, less precise.
This is why LHCb analyses typically focus on final states with charged particles
only, whereas \BaBar/ can afford to use final states with neutral particles.
We will come back to this point in the next section.

% Talk about run 1 and run 2 luminosity
LHCb collected data from 2010 to 2012 (Run 1), and 2015 to 2018 (Run 2).
An incomplete integrated luminosity (summing from 2010 to 2017) is
\SI{6.829}{fb^{-1}} \cite{LHCb-Facts:2019}.
% Talk about LS2 upgrade and LHCb's future
Currently, LHCb is shut down for an upgrade, which will greatly increase the
readout rate of the detector.
%We, the University of Maryland group, are actively participating in the upgrade.
%Specifically, we are designing data transmission system as well as power
%delivery system for the Upstream Tracker (UT) upgrade, which will replace the TT
%in Run 3.

%The current TT system limits the readout rate to \SI{1}{MHz}, due to the L0
%hardware trigger.
%The updated UT will have a \SI{40}{MHz} rate, which will make a fully
%software-based trigger possible \cite{LHCbCollaboration:2014tuj}.
%Another benefit is to reduce ghost tracks\footnote{
    %Ghost tracks are formed by linking VELO tracks with the wrong downstream
    %tracks.
%} by providing additional measurements between VELO and downstream trackers
%(currently IT, will be replaced by SciFi tracker) \cite{Parker:2017}.


\section{Review of previous measurements of \RDDst/}
In 2013, \BaBar/ reported a $3.4\sigma$ discrepancy in \RDDst/ from the
theoretical predictions \cite{Lees:2013rw}.
Since then, follow up measurements on the same semileptonic channel have been
performed by \BaBar/, BELLE, and LHCb, reporting various degrees of
discrepancies.
Currently, the world average of the measured \RDDst/ has a deviation of
$3.08\sigma$ with respect to SM prediction \cite{HFLAV:2019}, as shown
in \autoref{fig:rdrds_spring2019}.

\begin{figure}[ht]
    \centering
    \includegraphics[width=0.6\textwidth]{figs/rdrds_spring2019.pdf}
    \caption{
        World average of measured \RD/ and \RDst/, and SM predictions, as of
        spring 2019.
    }
    \label{fig:rdrds_spring2019}
\end{figure}

% B-factory measurements
As mentioned in \autoref{sec:babar}, $B$ factories, such as \BaBar/ and BELLE,
know the momenta of the $B \overline{B}$ system exactly.
By tagging and reconstructing the other $B$, known as $B_{tag}$, the momentum of
the signal $B_{sig}$, which decays semileptonically, can be inferred.

\BaBar/ and BELLE independently developed two types of tagging algorithms:
semileptonic tagging and hadronic tagging.
Semileptonic tagging finds $B_{tag}$ with the following decay:
$B^- \rightarrow l^- \bar{\nu}_l$, where $l$ is a $e$ or $\mu$.
This has the advantage of a larger branching ratio, thus more ($\approx 1\%$)
\Y4S/ events are tagged.
However, in this type of events, at least two neutrinos are presents, which
makes the reconstruction of $p_{B_{sig}}$ less precise \cite{Ciezarek:2017yzh}.

On the other hand, hadronic tagging searches over a very large number of
hadronic decay chains of $B$ for each \Y4S/ event, tagging the ones that match
one of the known modes.
This has a smaller tagging rate ($\approx 0.3$), but because $B_{tag}$ decays
hadronically, no neutrino is present in the final product.
This makes the reconstructed momentum of $B_{sig}$ very
precise \cite{Lees:2013uzd,Ciezarek:2017yzh}.

$B$ factory analyses also choose a particular decay mode of $\tau$:
$\tau^- \rightarrow l^- \nu_\tau \bar{\nu}_l$.
Thus the signal mode will have three neutrinos in the final product,
whereas the normalization mode only one.
By looking at the missing mass of the $B_{sig}$, defined as:
\begin{equation}
    m^2_{miss} \equiv \left(p_{B_{sig}} - p_{visible}\right)^2
\end{equation}
The signal, which has a non-zero $m^2_{miss}$, can be readily differentiated
from the normalization, which does have a $m^2_{miss} \approx 0$.

Non-$B \overline{B}$ background and misreconstructed events are suppressed by
requiring a tag, together with the requirement of having a $D^{(*)}$ meson and a
charged lepton on $B_{sig}$ \cite{Ciezarek:2017yzh}.

The main background remaining is semileptonic $B$ decays to $D^{**}$, which has
similar $m^2_{miss}$ distribution to the signal.
They can be differentiated from the signal by looking at the distribution of
$E^{*}_l$, the energy of the charged lepton in the $B$ rest
frame, because the background distribution extends to higher
energy \cite{Ciezarek:2017yzh}.

The 2013 \BaBar/ result used a Gaussian non-parametric kernel estimator in the
fitting process, because it allows for straightforward treading between variance
and bias \cite{Lees:2013uzd}, resulting in a better fit result.

% BELLE measurements
Other than the \BaBar/ result mentioned at the beginning of this section,
BELLE measured semileptonic $B$ decays with $\tau$ decays both
leptonically and hadronically.
Both semileptonic and hadronic tagging were used for the leptonic $\tau$ decay;
for hadronic $\tau$ decay, only hadronic tag was used.
The overall \RDDst/ deviation from the SM is about
$2\sigma$ \cite{Hirose:2017185}.

However, $B$ factories are limited by their data sample size, as discussed
in \autoref{sec:lhcb}.
This is the main advantage of LHCb.

% 2015 LHCb leptonic tau decay
The 2015 LHCb \RDst/ result, the first \RDst/ measurement at a hadron collider,
is $2.1\sigma$ larger than SM prediction \cite{LHCb:PhysRevLett.115.111803}.
This analysis is heavily inspired by the preview $B$ factory measurements, with
a few caveats:
The $B \overline{B}$ rest frame is not known in a hadron collider;
a significant contribution to the background due to the following decay mode:
$B \rightarrow D^* H_c X$, as these decay modes have similar $m^2_{miss}$ and
$E^{*}_l$ distributions very similar to the signal \cite{Ciezarek:2017yzh};
because the charge neutral particle resolution is not very good, a particular
decay mode, namely $\overline{B}^0 \rightarrow D^{*+} \tau^- \bar{\nu}_\tau$, is
chosen so that all visible final state particles are charged.

Without the $B \overline{B}$ rest frame, tagging algorithms used in the $B$
factory measurements cannot be applied.
This problem is solved by \emph{rest frame approximation}, a technique developed
by the analysis:
Assuming $\vec{(p_{B})_T}$ is unchanged, approximate $(p_{B})_z$ by the
following:
\begin{equation}
    (p_{B})_z = \frac{m_B}{m_{reco}} (p_{reco})_z
\end{equation}
with $m_B$ the known $B$ mass.
This is shown to have sufficient resolution \cite{LHCb:PhysRevLett.115.111803}.

The additional hadronic background is constrained by templates generated by
simulated data sample, with additional correction fitted from the
$D^{*+} \mu^- K^{\pm}$ control sample \cite{LHCb:PhysRevLett.115.111803}.
From \autoref{fig:babar_lhcb_fit_comparison}, we see that backgrounds of $B$
factories and LHCb have very different composition.

\begin{figure}[ht]
    \centering
    \includegraphics[width=0.85\textwidth]{figs/babar_lhcb_fit_comparison.pdf}
    \caption{
        Comparison between fitted data from \BaBar/ and LHCb.
    }
    \label{fig:babar_lhcb_fit_comparison}
\end{figure}

% 2018 LHCb hadronic tau decay: tau -> pi pi pi
The 2018 LHCb \RDst/ result is in agreement with the SM \cite{Aaij:2017deq}.
It used a different definition of \RDst/, renaming to $\mathcal{K}(D^{*-}))$:
\begin{equation}
    \mathcal{K}(D^{*-})) \equiv \frac{
        \mathcal{B}(B^0 \rightarrow D^{*-} \tau^+ \nu_\tau)
    }{
        \mathcal{B}(B^0 \rightarrow D^{*-} 3 \pi)
    }
\end{equation}

It also took a different approach to separate signal from normalization:
By using a different $t$ decay mode $\tau^- \rightarrow \pi^+ \pi^- \pi^+$,
the $\tau$ decay vertex can be reconstructed from the three charged pions.
Since $\tau$ flies a certain distance before it decays, its decay vertex is
separated from its generation vertex.
Thus, the signal events have a different topology than the normalization, of
which all final state particles point to the same vertex \cite{Aaij:2017deq}.

By requiring a clear separation between the two vertices, the prompt background
($D^* \pi \pi \pi X$) can be rejected.
The remaining double-charm background has the same topology as the signal.
A template derived from simulations is used to reject such
background \cite{Aaij:2017deq}.

\section{Outlook for LHCb Run 2 \RDDst/ measurements}
% Expected improvements
% Remember: Run 2 supposedly has better pile-ups.
% Current progress

\PRLrule
\printbibliography
\end{document}
