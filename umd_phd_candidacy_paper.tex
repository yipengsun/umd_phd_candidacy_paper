\documentclass[10pt]{article}

%%%%%%%%%%%%%%%%%%%%%%%%%
% Package configuration %
%%%%%%%%%%%%%%%%%%%%%%%%%

% Page margin
\usepackage[margin=1in]{geometry}

% Support for bold small cap font
\usepackage[tuenc]{fontspec}%for lualatex case
\setmainfont{CMU Serif}

% Better typesetting quality
% also used to customize section style
\usepackage{microtype}

% Redefine section, subsection styles
\usepackage[compact,center,explicit]{titlesec}
\usepackage{textcase}
\titleformat{\section}{\scshape\lsstyle\normalsize\filcenter}
    {\thesection}{1em}{\textls{\MakeTextUppercase{#1}}}
\titleformat{\subsection}{\normalfont\small\bfseries\filcenter}
    {\thesubsection}{1em}{#1}

% Setup link
\usepackage{hyperref}
\hypersetup{colorlinks,breaklinks,citecolor=blue}

% Set up biblatex database
\usepackage[
    %style=authortitle-tcomp,
    %firstinits=true,
    %backref=true,
    natbib=true,
    backend=biber,
    % NOTE: This options ensures that no automatic et al. is generated
    maxbibnames=99,
    % NOTE: This option must be enabled with 'babel' package
    useprefix=false
]{biblatex}
\addbibresource{umd_phd_candidacy_paper.bib}

%%%%%%%%%%%%%%%%%
% User settings %
%%%%%%%%%%%%%%%%%

% User-defined variables
\def\BaBar/{\textsc{BaBar}}

% Title info
\title{Review on Testing lepton flavor universality in semileptonic channels}
\author{Yipeng Sun}
\date{\today}

\begin{document}
\maketitle

\section{Introduction}
What is LFUV? (FU physics, FUV physics)
Why semileptonic decays (good thing about form factor)?
A historical review, starting from 2012 BaBar;
Talk about Phoebe's run 1 analyses.
Talk about what we are doing.
Talk about expected improvements and drawbacks of our upcoming analysis compared
to run 1.

\section{Theory}
\subsection{Lepton flavor universality}

\subsection{Advantages of semileptonic channel decays}

\subsection{Higgs mechanism}

\subsection{2-Higgs doublet models (2HDM)}
% type-II model claimed to be excluded; type-III (very similar to type-II) still
% alive.

\subsection{Leptoquark models}


\section{Review of colliders/detectors for $B$-physics}
One may ask: Why are we so interested in $B$ mesons?
% Initially B factories are meant for CP violation detection.
Initially, B factories, such as \BaBar/, was primarily constructed to do
precision measurements on CP violation of \cite{Luth:1994}.


% Talk about why B-bound states/mesons are also good for testing LFUV.
$B$ mesons provide a good testing bed for new physics:
They have enough energy to decay into all 3 flavors of leptons;
at the same time, they are not too heavy to produce.

All measurements start from $b\bar{b}$ bound states.
Electron-position colliders\footnote{
    Such as PEP-II, the collider for the \BaBar/ experiment.
} produce $\Upsilon_{4s}$ via electroweak interaction
Hadron colliders\footnote{
    Such as Large Hadron Collider (LHC) for the LHCb experiment.
} produce $\Upsilon_{4s}$ via strong interaction.


% Talk about PEP-II and its asymmetrical beam energies
PEP-II is an asymmetrical $e^- e^+$ collider at SLAC.
In PEP-II, $B$ mesons are produced primarily in the following process:
$e^- e^+ \rightarrow \Y4S/ \rightarrow B \bar{B}$, with
$e^-$ and $e^+$ beams tuned at different energies,
such that the invariant mass is at the \Y4S/ resonance (\SI{10.58}{GeV}),
and the momentum of the \Y4S/ in the lab frame
non-zero \cite{Harrison:1998yr}.

Producing at \Y4S/ peak leads to near-exclusive $B \bar{B}$ pair meson
production, reducing combinatorial background.
Also, since the momenta of $e^- e^+$ is known, with the reconstruction of the
momentum of one $B$ meson ($B_{tag}$), the momentum of the other $B$
($B_{sig}$) can be calculated as
\begin{equation}
    p_{B_{sig}} = p_{e^-e^+} - p_{B_{tag}}.
\end{equation}
Later we will see that this makes identifying events that have more than one
missing particle easier.

% Talk about subdetectors
\BaBar/ is a nearly $4\pi$ spetrometer (shown in
\autoref{fig:babar_detector_view}) that consists of five subdetectors.
From inside out:
Silicon Vertex Tracker (SVT) and Drift Chamber (DCH), which measure the momenta
and angles of charged particles with the help of a \SI{1.5}{T}.
Detector of Internally Reflected Cerenkov radiation (DIRC), together with SVT
and DCH, identifies charged particles of different masses by Cerenkov
ring-imaging and ionization energy loss of these particles.
Cesium Iodide Electromagnetic Calorimeter (EMC) measures energy and
position of electromagnetic showers generated by electrons and photons.
A superconducting solenoid with a \SI{1.5}{T} magnetic field surrounding the
EMC is part of the tracking and particle identification system which identifies
muons and some neutral hadrons together with Instrumented Flux Return
(IFR) \cite{Lees:2013uzd}.

\begin{figure}[ht]
    \centering
    \includegraphics[width=0.7\textwidth]{figs/babar_detector_view.pdf}
    \caption{
        View of the \BaBar/ detector.
        Extracted from \cite{Boutigny:1995ib}.
    }
    \label{fig:babar_detector_view}
\end{figure}

% Talk about BaBar being 4 pi
At $B$ factories, $b \bar{b}$ are produced at all angles with
non-negligible probability \cite{Boutigny:1995ib,McGregor:2008ek}, thus the
detector needs to cover almost all solid angles (a $4\pi$ detector).
Indeed, \BaBar/ has tracking coverage of 0.92, namely 92\% of the $4\pi$ solid
angle.

% Talk about tracking and calorimeters
Measurement of time dependent CP violation in neutral $B$ decays requires
excellent vertex resolution and tracking, because the two $B$ mesons produced by
\Y4S/ must be reliably separated.
\BaBar/ has excellent tracking for charged particles, and sufficient spatial
and energy resolution in the electromagnetic calorimeter to reconstruct the
momenta of neutral particles \cite{Bauer:2005} with good precision.

\input{include/detectors/01-belle.tex}
\input{include/detectors/02-lhcb.tex}


\section{Review of previous measurements}

\subsection{2013 \BaBar/ $R(D^{(*)})$}
% 2015 BELLE (also hadronic tag)
% BELLE semileptonic tags

\subsection{2016 LHCb $R(D^{*})$}

\subsection{recent measurements from LHCb (forgot the decay channels)}
%2018 LHCb hadronic decay of Tau -> 3* Pi R(D*)


\section{Outlook for LHCb Run 2 measurements}
%\subsection{Current progress}

%\subsection{Expected improvements}

%\subsection{possible drawbacks}
% Remember: Run 2 supposedly has better pile-ups.

\vspace{5em}
\printbibliography
\end{document}
