\documentclass[10pt]{article}

%%%%%%%%%%%%%%%%%%%%%%%%%
% Package configuration %
%%%%%%%%%%%%%%%%%%%%%%%%%

% Page margin
\usepackage[margin=1in]{geometry}

% Support for bold small cap font
\usepackage[tuenc]{fontspec}
\setmainfont[
    Path=./fonts/,
    Extension=.otf,
    BoldFont=cmu-serif-bold,
    BoldItalicFont=cmu-serif-bold-italic,
    ItalicFont=cmu-serif-italic,
]{cmu-serif}

% Better typesetting quality
\usepackage{microtype}

% Math
\usepackage{amsmath}

% SI-units
\usepackage{siunitx}

% Redefine section, subsection styles
\usepackage[compact,center,explicit]{titlesec}
\usepackage{textcase}
\titleformat{\section}{\scshape\lsstyle\normalsize\filcenter}
    {\thesection}{1em}{\textls{\MakeTextUppercase{#1}}}
\titleformat{\subsection}{\normalfont\small\bfseries\filcenter}
    {\thesubsection}{1em}{#1}

% PRL-style horizontal rule
\usepackage{graphicx,amssymb}
\newcommand{\PRLrule}{
    \bigskip
    \noindent\makebox[\linewidth]{
        \resizebox{0.3333\linewidth}{1pt}{$\blacklozenge$}
    }
    \bigskip
}

% Set up link
\usepackage{hyperref}
\hypersetup{colorlinks,breaklinks,citecolor=blue}

% Set up author affiliation
\usepackage[affil-it]{authblk}

% Set up biblatex database
\usepackage[
    %style=phys,
    giveninits=true,
    %backref=true,
    natbib=true,
    backend=biber,
    doi=true,
    % Sort by the order of citation
    sorting=none,
    % This options ensures that no automatic et al. is generated
    %maxbibnames=99,
    % This option must be enabled with 'babel' package
    useprefix=false
]{biblatex}
\addbibresource{umd_phd_candidacy_paper.bib}


%%%%%%%%%%%%%%%%%
% User settings %
%%%%%%%%%%%%%%%%%

% User-defined variables
\def\BaBar/{\textsc{BaBar}}
\def\Y4S/{\ensuremath{\Upsilon(\text{4S})}}

% Title info
\title{Review on Testing lepton flavor universality in semileptonic channels}
\author{Yipeng Sun}
\affil{Department of Physics, University of Maryland}
\date{\today}


%%%%%%%%%%%
% Content %
%%%%%%%%%%%

\begin{document}
\maketitle

\begin{abstract}
    What is LFUV? (FU physics, FUV physics)
    Why semileptonic decays (good thing about form factor)?
    A historical review, starting from 2012 BaBar;
    Talk about Phoebe's run 1 analyses.
    Talk about what we are doing.
    Talk about expected improvements and drawbacks of our upcoming analysis compared
    to run 1.
\end{abstract}

\section{Theory}
% Lepton flavor universality (LFU) and LFU violation (LFUV)
\input{include/theories/00-lepton_flavor_universality_violation.tex}

\subsection{Advantages of semileptonic channel decays}

\subsection{Higgs mechanism}

\subsection{2-Higgs doublet models (2HDM)}
% type-II model claimed to be excluded; type-III (very similar to type-II) still
% alive.

\subsection{Leptoquark models}


\section{Review for detectors/colliders used in testing LFU}
The LFU has been tested in many precision measurements.
The majority of them involve the difference between the decay rates of $B$
mesons to $\tau$ and $\mu$.

% Why not J/psi (c cbar, mass ~ 3.10 GeV, mass of tau ~ 1.77 GeV)?
One may ask: Why are we so focused on $b$ quark bound states and mesons?
After all, $J/\psi$, a $c\bar{c}$ bound state, is already kinematically allowed
to decay into all three generations of leptons.
The main reason is:
% Initially B factories are meant for CP violation detection.
Initially, detectors of $B$ factories, such as \BaBar/ at PEP-II, were
primarily constructed to for precision measurements on CP violation of $B^0$,
for SM predicts ``large, calculable'' CP violation in the decay of these mesons
\cite{Luth:1994}.
But these detectors proved to be advantageous in the testing of LFU:
These measurements have very similar requirements on the
detector \cite{Boutigny:1995ib}.
Thus, testing of LFU is often part of the secondary goals of these
experiments \cite{Luth:1994}.

In this section, I will review \BaBar/ detector at the PEP-II collider, and LHCb
at the Large Hadron Collider (LHC)---both have conducted various tests on LFU.
These detectors/colliders are representative of the detectors for
electron-position colliders and hadron colliders.

% Talk about PEP-II and its asymmetrical beam energies
PEP-II is an asymmetrical $e^- e^+$ collider at SLAC.
In PEP-II, $B$ mesons are produced primarily in the following process:
$e^- e^+ \rightarrow \Y4S/ \rightarrow B \bar{B}$, with
$e^-$ and $e^+$ beams tuned at different energies,
such that the invariant mass is at the \Y4S/ resonance (\SI{10.58}{GeV}),
and the momentum of the \Y4S/ in the lab frame
non-zero \cite{Harrison:1998yr}.

Producing at \Y4S/ peak leads to near-exclusive $B \bar{B}$ pair meson
production, reducing combinatorial background.
Also, since the momenta of $e^- e^+$ is known, with the reconstruction of the
momentum of one $B$ meson ($B_{tag}$), the momentum of the other $B$
($B_{sig}$) can be calculated as
\begin{equation}
    p_{B_{sig}} = p_{e^-e^+} - p_{B_{tag}}.
\end{equation}
Later we will see that this makes identifying events that have more than one
missing particle easier.

% Talk about subdetectors
\BaBar/ is a nearly $4\pi$ spetrometer (shown in
\autoref{fig:babar_detector_view}) that consists of five subdetectors.
From inside out:
Silicon Vertex Tracker (SVT) and Drift Chamber (DCH), which measure the momenta
and angles of charged particles with the help of a \SI{1.5}{T}.
Detector of Internally Reflected Cerenkov radiation (DIRC), together with SVT
and DCH, identifies charged particles of different masses by Cerenkov
ring-imaging and ionization energy loss of these particles.
Cesium Iodide Electromagnetic Calorimeter (EMC) measures energy and
position of electromagnetic showers generated by electrons and photons.
A superconducting solenoid with a \SI{1.5}{T} magnetic field surrounding the
EMC is part of the tracking and particle identification system which identifies
muons and some neutral hadrons together with Instrumented Flux Return
(IFR) \cite{Lees:2013uzd}.

\begin{figure}[ht]
    \centering
    \includegraphics[width=0.7\textwidth]{figs/babar_detector_view.pdf}
    \caption{
        View of the \BaBar/ detector.
        Extracted from \cite{Boutigny:1995ib}.
    }
    \label{fig:babar_detector_view}
\end{figure}

% Talk about BaBar being 4 pi
At $B$ factories, $b \bar{b}$ are produced at all angles with
non-negligible probability \cite{Boutigny:1995ib,McGregor:2008ek}, thus the
detector needs to cover almost all solid angles (a $4\pi$ detector).
Indeed, \BaBar/ has tracking coverage of 0.92, namely 92\% of the $4\pi$ solid
angle.

% Talk about tracking and calorimeters
Measurement of time dependent CP violation in neutral $B$ decays requires
excellent vertex resolution and tracking, because the two $B$ mesons produced by
\Y4S/ must be reliably separated.
\BaBar/ has excellent tracking for charged particles, and sufficient spatial
and energy resolution in the electromagnetic calorimeter to reconstruct the
momenta of neutral particles \cite{Bauer:2005} with good precision.

%The LHC is a $pp$ collider.
% Talk about the LHC being a hadron collider and the difficulties associated
% with it
Unlike electron, proton is a composite particle, made of $u, u, d$ quarks.
From the parton distribution functions of LHCb\footnote{
    Number density to find fraction of the momentum (denoted as $x$) at certain
    squared energy scale $Q^2$.
}, we see that there are plenty of other elementary particles, such as gluons,
that can participate in the collision.
These particles may carry varying portion of the total
momentum \cite{Ball:2014uwa}.

Effectively, partons, including gluons, are being collided.
Due to the unavoidable strong actions, many unwanted particles will be
generated---Comparing to \BaBar/, one notable addition of LHCb is the triggering
system, which is used to filter out uninteresting events, reducing the readout
rate \cite{LHCb:2008}.
Also, since we do not know the precise fraction of momentum carried by
interactive partons\footnote{
Again, these are characterized by parton distribution functions}, the $B$ meson
rest frame is not readily known.

It is obvious that $e^- e^+$ colliders provide a \emph{much} cleaner background.
However, the LHC generates much more $b\bar{b}$\footnote{
    Note that this is \emph{not} necessarily \Y4S/.
} events compared to \BaBar/, due to a much larger cross section.
At \SI{13}{TeV}, the measured cross section at LHCb\footnote{
    For $2 < \eta < 5$ only, since this is the LHCb acceptance range.
} is $144 \pm 1 \pm 21$~\si{\mu b} \cite{Aaij:2016avz}.
% FIXME: Is the calculation for BaBar cross section legal?
Use the integrated luminosity and total number of $B\overline{B}$ events
contained in the on-resonance \Y4S/ sample, we compute the $b\bar{b}$ cross
section of \BaBar/ to be $\approx 1.09$~\si{nb}, which is much smaller than that
of the LHCb.

% Talk about subdetectors
LHCb, a single-arm spectrometer, is one of the four large experiments at the
LHC.
Its constituent subdetectors, from closest to farthest from the collision point,
are shown in \autoref{fig:lhcb_detector_view}:
The Vertex Locator (VELO) provides precise measurements of track coordinates
close to the collision point.
Two Ring Imaging Cerencov counters (RICH1, RICH2) provide particle
identification for charged particles over a wide range of momentum.
Tracker Turicensis (TT) and Inner Tracker (IT) provide additional tracking for
charged particles.
The Outer Tracker (OT) is used for tracking, as well as measures the momentum
of charged particles.
The calorimeters (ECAL and HCAL) have a first-level (L0) trigger to select
hadron, electron, and photon candidates based on their transverse momentum
$p_T$;
they also provide identification for the particles listed above;
finally, they provide energy and position measurements for these particles.
The Muon system (M1-5) is farthest from the collision point;
it provides L0 high $p_T$ muon trigger, and a high-level trigger (HLT) for muon
identification \cite{LHCb:2008}.

\begin{figure}[ht]
    \centering
    \includegraphics[width=0.7\textwidth]{figs/lhcb_detector_view.pdf}
    \caption{
        View of the LHCb detector.
        Vertex Locator (VELO) is closest to collision point.
    }
    \label{fig:lhcb_detector_view}
\end{figure}

% Talk about LHCb being forward-only
An interesting design choice is the geometry of the LHCb detector:
Instead of being a barrel $4\pi$ detector, it is forward-only.
This is because at high energies, $b\bar{b}$ is mostly produced in the forward
and backward direction.
The LHCb design is a very cost-effective way to construct a detector at the LHC
dedicated for $B$ physics \cite{LHCb:2008}.

% Talk about tracking
LHCb has a very good vertexing and tracking system, some of the subdetectors has
a better resolution, even compared to \BaBar/;
but its calorimeters are mediocre \cite{LHCb:2008,Guz:2017}, which makes the
reconstruction of charge-neutral particles, such as $\pi^0$, less precise.
This is why LHCb analyses typically focus on final states with charged particles
only, whereas \BaBar/ can afford to use final states with neutral particles.
We will come back to this point in the next section.

% Talk about run 1 and run 2 luminosity
LHCb collected data from 2010 to 2012 (Run 1), and 2015 to 2018 (Run 2).
An incomplete integrated luminosity (summing from 2010 to 2017) is
\SI{6.829}{fb^{-1}} \cite{LHCb-Facts:2019}.
% Talk about LS2 upgrade and LHCb's future
Currently, LHCb is shut down for an upgrade, which will greatly increase the
readout rate of the detector.
%We, the University of Maryland group, are actively participating in the upgrade.
%Specifically, we are designing data transmission system as well as power
%delivery system for the Upstream Tracker (UT) upgrade, which will replace the TT
%in Run 3.

%The current TT system limits the readout rate to \SI{1}{MHz}, due to the L0
%hardware trigger.
%The updated UT will have a \SI{40}{MHz} rate, which will make a fully
%software-based trigger possible \cite{LHCbCollaboration:2014tuj}.
%Another benefit is to reduce ghost tracks\footnote{
    %Ghost tracks are formed by linking VELO tracks with the wrong downstream
    %tracks.
%} by providing additional measurements between VELO and downstream trackers
%(currently IT, will be replaced by SciFi tracker) \cite{Parker:2017}.



\section{Review of previous measurements}

\subsection{2013 \BaBar/ $R(D^{(*)})$}
% 2015 BELLE (also hadronic tag)
% BELLE semileptonic tags

\subsection{2016 LHCb $R(D^{*})$}

\subsection{recent measurements from LHCb (forgot the decay channels)}
%2018 LHCb hadronic decay of Tau -> 3* Pi R(D*)


\section{Outlook for LHCb Run 2 $R(D^{(*)})$ measurements}
%\subsection{Current progress}

%\subsection{Expected improvements}

%\subsection{possible drawbacks}
% Remember: Run 2 supposedly has better pile-ups.

\PRLrule
\printbibliography
\end{document}